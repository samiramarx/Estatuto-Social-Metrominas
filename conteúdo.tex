\documentclass[a4paper,11pt]{report} 
\usepackage[utf8]{inputenc} % Permite caracteres acentuados

\usepackage{fontspec} %Altera para fonte Metrominas
\setmainfont{tw-cen-mt.ttf}[
BoldFont = tw-cen-mt-bold.ttf ,
ItalicFont = tw-cen-mt-italic.ttf,
BoldItalicFont = tw-cen-mt-bold-italic.ttf ]

\setlength{\parindent}{0pt} % Retira indentação dos parágrafos

\usepackage{enumitem}
\renewcommand*{\theenumi}{\thesection.\arabic{enumi}} % Permite numeração continuada a partir dos capítulos e seções.
\renewcommand*{\theenumii}{\theenumi.\arabic{enumii}} % Permite numeração continuada a partir dos capítulos e seções.
\usepackage{hyperref} % Utilizado no sistema de referência cruzada - vide arquivo "cross-reference.txt" 

\usepackage[portuguese]{babel}
\usepackage{fancyhdr}
\pagestyle{fancy}
\setlength{\headheight}{26pt}
\rhead{Trem Metropolitano de Belo Horizonte S.A.}
\lhead{}

\setlength\parskip{2ex} % Inclui espaço entre parágrafos

\usepackage{titlesec} % Número e nome dos capítulos e seções na mesma linha
\titleformat{\chapter}[hang]
{\normalfont\normalsize\bfseries}{CAPÍTULO \thechapter.}{0.5em}{}
\titlespacing*{\chapter}{0pt}{10pt}{10pt}

\titleformat{\section}[hang]
{\normalfont\normalsize\bfseries}{Seção \thesection.}{0.5em}{}

\usepackage{titletoc}%
\titlecontents{chapter}% <section-type>
  [0pt]% <left>
  {\bfseries}% <above-code>
  {\chaptername\ \thecontentslabel.\quad}% <numbered-entry-format>
  {}% <numberless-entry-format>
  {\hfill\contentspage}% <filler-page-format>


\renewcommand{\thechapter}{\Roman{chapter}} %Altera numeração de capítulos para números romanos
\renewcommand{\thesection}{\Roman{section}} %Altera numeração de seções para números romanos
\renewcommand{\thesubsection}{\Roman{subsection}} %Altera numeração de subseções para números romanos

%\renewcommand*\thesection{\arabic{section}} %Altera numeração de seções para não estarem vinculadas aos números dos capítulos

% para evitar quebra de página após capítulo
\usepackage{etoolbox}
\makeatletter
\patchcmd{\chapter}{\if@openright\cleardoublepage\else\clearpage\fi}{}{}{}
\makeatother

%-----------------------------------------------------------
%	PREENCHIMENTO DE DADOS
%-----------------------------------------------------------

% Dados do Ato
\newcommand{\NumeroAto}{$\bullet$}
\newcommand{\DataAssinatura}{\today}
\newcommand{\Signatario}{\textit{[Nome Signatário]}}
\newcommand{\CargoSignatario}{Subsecretário de Regulação Transportes}


%------------------------------------------------------------
\begin{document}
%------------------------------------------------------------

\begin{center}
ESTATUTO SOCIAL DA COMPANHIA TREM METROPOLITANO DE BELO HORIZONTE S/A

CNPJ: 03.919.139/0001-21 - NIRE 3150021774-8
\end{center}

\chapter{DA DENOMINAÇÃO, SEDE, OBJETO E DURAÇÃO}

\begin{enumerate}[resume, label=Art. \arabic*]

\item Trem Metropolitano de Belo Horizonte S.A. é uma Empresa Pública, sob a forma de sociedade anônima vinculada à Secretaria de Transportes e Obras Públicas autorizada pela Lei Estadual nº 12.590, de 25 de julho de 1997, e rege-se por este Estatuto e pela legislação em vigor.

\item A Companhia tem sua sede, foro e administração na cidade de Belo Horizonte, Estado de Minas Gerais.

Parágrafo único. Sempre que o interesse da Companhia o exigir, a Companhia poderá, a critério e por deliberação do Conselho de Administração, abrir filiais, agências, sucursais, ou escritórios no País ou no exterior, mediante aprovação na Assembleia Geral.

\item O prazo de duração da Companhia é indeterminado.

\item A Companhia tem como objeto a implantação, construção, operação, manutenção e exploração do transporte metroviário e ferroviário de passageiros na Região Metropolitana de Belo Horizonte, bem como todas as atividades conexas, tais como planejamento, projetos, construção e implantação de equipamentos para execução destes serviços; a exploração de seus bens e direitos patrimoniais, comercialização de marcas ou insígnias e de espaço para propaganda, a prestação de serviços de manutenção de equipamentos e, ainda, a participação em outras empresas com objeto social correlato.

Parágrafo único. O desenvolvimento do planejamento, projetos, construções e implantações de instalações e equipamentos deverá estar em perfeita consonância com a política de transporte e desenvolvimento urbano da Região Metropolitana de Belo Horizonte e dos municípios atendidos. Para tal, a Companhia deverá articular-se com outros, com organizações de direito privado e entidades de direito público metropolitano, estaduais ou municipais, para o cumprimento de seus objetivos, podendo para isto celebrar convênios de cooperação técnica, operacional e financeira.

\item A Companhia poderá subscrever ações de Empresas das quais o Poder Público tenha o controle acionário e cujas atividades se relacionem com transpores, tráfego, trânsito e sistema viário da Região Metropolitana de Belo Horizonte, celebrar contratos, convênios e constituir consórcios com pessoas jurídicas, de direito privado, contrair empréstimos e contratar financiamentos, bem como promover serviços de desapropriação, mediante prévia declaração de utilidade pública ou de interesse social, e estabelecer servidões administrativas, mediante prévio ato declaratório.

\item Todos os serviços prestados pela Companhia serão obrigatoriamente remunerados.

\chapter{CAPITAL SOCIAL}

\item O Capital Social da Companhia é de R\$ 500.000,00 (quinhentos mil reais) representado por 500.000 (quinhentos mil) ações ordinárias nominativas no valor de R\$1,00 (um real) cada, podendo ser emitidos títulos múltiplos de ações.

\begin{enumerate}[label= \S \arabic*]
    \item A Capitalização da reserva de capital, resultante da correção monetária, do capital realizado será efetivada sem modficações do número de ações emitidas.
    \item As ações representativas do aumento do capital social, mediante subscrição serão sempre ordinárias nominativas.
\end{enumerate}

\item O Estado de Minas Gerais manterá a titularidade do percentual das ações com direito a voto que lhe assegure o controle acionário da Companhia, sendo as demais ações de titularidade dos municípios da Região Metropolitana que as subscrevem, de acordo com o art. 6º da lei 12.590, de 25 de julho de 1997.

\item As ações são indivisíveis em relação à Companhia, cabendo a cada uma o direito de voto nas deliberações das Assembléias Gerais.

Parágrafo único. As ações, os títulos e cautelas serão sempre assinadas pelo Diretor-Presidente, juntamente com qualquer outro Diretor.

\item A integralização das ações subscritas será feita, nas condições e prazos estabelecidos pela Assembléia Geral, com observância das disposições aplicadas, segundo a Lei nº 6.404/76.

\chapter{ÓRGÃOS DA COMPANHIA}

\item São órgãos da Companhia:

\begin{enumerate}[label=\roman*.]
    \item a Assembléia Geral
    \item o Conselho de Administração
    \item a Diretoria Executiva
    \item o Conselho Fiscal
\end{enumerate}

\item A Companhia será administrada por um Conselho de Admnistração com atribuições deliberativas e normativas e por uma Diretoria com atribuições executivas.

\begin{enumerate}[label= \S \arabic*]
    \item O mandato dos administradores é de 3 (três) anos, permitida a reeleição.
    \item Os conselheiros e diretores serão investidos em seus  cargos mediante assinatura de termo de posse lavrada no livro de atas do Conselho de Administração e da Diretoria, respectivamente, constituindo requisito prévio e essencial à investidura em tais cargosa apresentação da declaração de bens, pelo eleito, a qual será igualmente exigida ao término de seus mandatos.
    \item o prazo de gestão dos membros do Conselho de Administração e da Diretoria estender-se-á até a investidura dos novos administradores.
\end{enumerate}


\section{DA ASSEMBLÉIA GERAL}

\item A Assembléia Geral reunir-se-á, ordinariamente, no primeiro quadrimestre de cada ano e, extraordinariamente, sempre que convocada de acordo com a lei e com este Estatuto.

\item A Assembléia Geral será convocada pelo Conselho de Administração, mediante anúncios publicados pela imprensa, dos quais deverão constar a "Ordem do Dia", o dia, a hora e o local da reunião.

\begin{enumerate}[label= \S \arabic*]
    \item A Assembléia Geral poderá também ser convocada pelo Conselho Fiscal ou pelos acionistas, observadas as disposições legais aplicáveis.
    \item A Assembléia Geral será instalada e presidida pelo acionista majoritário ou seu representante legal, sendo o secretário escolhido dentre os representantes dos acionistas presentes.
\end{enumerate}

\item À Assembléia Geral Ordinária caberá:

\begin{enumerate}[label=\roman*.]
    \item tomar as contas dos administradores;
    \item examinar, discutir e votar as demonstrações financeiras;
    \item deliberar sobre a destinação do lucro líquido do exercício e a distribuição dos dividendos;
    eleger e destituir os membros do Conselho de Administração e do Conselho Fiscal;
    \item aprovar a correção da expressão monetária do capital social.
\end{enumerate}

\item À Assembléia Geral Extraordinária caberá:

\begin{enumerate}[label=\roman*.]
    \item a reforma deste Estatuto Social;
    \item deliberar sobre a proposta de aumento de capital social;
    \item fixar o montante de recursos para a manutenção dos diretores e membros dos conselhos;
    \item deliberar sobre assuntos não enunciados nos artigos anteriores.
\end{enumerate}

\section{CONSELHO DE ADMINISTRAÇÃO}

\item O Conselho de Administração será composto por 7 (sete) mebros, sendo presidido por um deles, todos pessoas naturais residentes no País eleitos pelo prazo de 3 (três) anos pela Asembléia Geral, podendo ser reeleitos.

Parágrafo único. Entre os membros do Conselho de Administração 4 (quatro) serão indicados pelo Estado de Minas Gerais, 2 (dois) pelo Município de Belo Horizonte e 1 (um) pelo Muncípio de Contagem.

\item O Conselho de Administração terá um Presidente, que será o Secretário de Transportes e Obras Públicas, e um Vice-Presidente, escolhido por seus pares, por maioria de votos, na primeira reunião após sua posse.

\begin{enumerate}[label= \S \arabic*]
    \item O Presidente do Conselho de Administração será submetido, em suas faltas, ausências e impedimentos, pelo Vice-Presidente.
    \item Ocorrendo vacância, impedimentos ou licença, os membros do Conselho de Administração elegerão os seus substitutos "ad referendum"da Assembléia Geral, os quais exercerão o restante do mandato de seus substitutos.
    \item No caso de vacância da maiora dos membros do Conselho de Administração caberá à Diretoria Executiva a convocação da Assembléia Geral.
\end{enumerate}

\item O Conselho de Administração reunir-se-á, ordinariamente, no mínimo uma vez por mês e, extraordinariamente, sempre que convocado pelo seu presidente, pelo Diretor-Presidente ou pela maioria de seus membros.

\begin{enumerate}[label= \S \arabic*]
    \item As reuniões do Conselho de Administração somente se realizarão com a presença da maioria de seus membros, sendo as deliberações tomadas por maioria dos votos dos presentes.
    \item Caberá ao Presidente do Conselho, além do seu voto, o voto de qualidade, em caso de empate.
\end{enumerate}

\item As deliberações do Conselho de Administração constarão de atas lavradas em livro próprio e serão assinadas pelos conselheiros presentes.

\begin{enumerate}[label= \S \arabic*]
    \item Serão arquivadas na Junta Comercial e publicadas as atas das reuniões do Conselho de Administração que contiverem deliberações destinadas a produzir efeitos permanente a terceiros.
    \item Compete ao Presidente do Conselho de Administração informar à Diretoria Executiva e à Assembléia Geral, conforme o caso, as deliberações tomadas em suas reuniões. Todas as notificações endereçadas ao Conselho de Administração deverão ser enviadas ao seu Presidente.
\end{enumerate}

\item Compete ao Conselho de Administração, além das suas atribuições que lhe são conferidas por lei e por Estatuto:

\begin{enumerate}[label=\roman*.]
    \item fixar a orientação geral dos negócios da Companhia;
    \item eleger e destituir os diretores da companhia e fixar-lhes as atribuições, observado o que a respeito dispuser o estatuto;
    \item fiscalizar a gestão dos diretores, examinar, a qualquer tempo, os livros e papéis da companhia, solicitar informações sobre contratos celebrados ou em via de celebração, e quaisquer outros atos.
    \item convocar a Assmebléia Geral quando julgar conveniente, ou quando a lei o determinar;
    \item manifestar-se sobre o relatório de administração e as contas da diretoria;
    \item, manifestar-se previamente sobre atos ou contratos, quando o estatuto assim o exigir;
    \item deliberar, quando autorizado pelo estatuto, sobre a emissão de ações ou de bônus de subscrição;
    \item autorizar, se o estatuto não dispuser em contrário, a alienação de bens do ativo permanente, a constituição de ônus reais e a prestação de garantias a obrigações de terceiros;
    \item escolher e destituir os auditores independentes, se houver;
    
    Parágrafo único. Serão arquivados no registro do comércio e publicadas as atas das reuniões do conselho de administração que contiverem deliberação destinada a produzir efeitos perante terceiros.
    
\end{enumerate}

\item Os membros do Conselho de Administração farão jus a remuneração simbólica fixada pela Assembléia Geral.

\item O Conselho de Administração estabelecerá as metas de atuação da Companhia, seguindo os modelos mais modernos da economia, de forma a promover a conduçãodos negóciosda sociedade de maneira empresarial.

\begin{enumerate}[label= \S \arabic*]
    \item A Companhia manterá padrões de gestão empresarial, tanto na área administrativa quanto na operacional de acordo com indicadores de desempenho definidos por ato do Secretário de Estado de Transportes e Obras Públicas - SETOP.
\end{enumerate}

\section{DIRETORIA EXECUTIVA}

\item A Companhia terá uma Diretoria Executiva constituída de até 5 (cinco) profissionais de nível superior de reconhecida capacidade técnica, eleitos pelo Conselho de Administração com mandato de 3 (três) anos, permitida a reeleição.

\begin{enumerate}[label= \S \arabic*]
    \item A Diretoria Executiva terá no máximo a seguinte composição:
    \begin{enumerate}[label=\roman*.]
    \item Um Diretor Presidente, eleito entre os membros do Conselho de Administração e quatro diretores executivos.
    \end{enumerate}
    \item Na hipótese de ausência ou impedimento temporário de qualquer de seus membros, as respectivas atribuições serão desempenhadas segundo o Diretor Presidente.
    \item No caso de vacância do cargo de Diretor Presidente, um Diretor, indicado previamente pelo Conselho, o substituirá, até a reunião do Conselho de Administração deliberar sobre o provimento do cargo, acumulando os dois cargos e optando por uma única das remunerações.
    \item No impedimento ocasional do Diretor Presidente, este escolherá o seu substituto enre os demais membros da Diretoria Executiva, que optará por uma única das remunerações correspondentes.
    \item Vagando o cargo de Diretor, poderá o Conselho de Administração indicar novo Diretor, pelo prazo que restava ao subistituto, ou alternativamente deixá-lo vago designando outro Diretor para responder pela Diretoria vaga.
    \item Nos impedimentos ocasionais ou temporários do Diretor, o Diretor Presidente designar-lhe-á substituto, dentre os demais Diretores eleitos pelo Conselho de Administração, que não acumulará os vencimentos respectivos.
    \item Os membros da Diretoria Executiva permanecerão no exercício de seus cargos até que seus substitutos sejam empossados.
    \item Dentre os Diretores eleitos pelo Conselho de Administração, 3 (três) serão indicados pelo Estado de Minas Gerais, 1 (um) pelo Município de Belo Horizonte e 1 (um) pelo Município de Contagem.
\end{enumerate}

\item Os honorários dos membros da Diretora Executiva, e seus reajustes, serão fixados pela Assembléia Geral.
\item A Diretoria Executiva reunir-se-á uma vez por mês, ou sempre que convocada pelo Diretor Presidente, com a presença da maioria de seus membros.

    \begin{enumerate}[label= \S \arabic*]
    \item As reuniões da Diretoria Executiva serão convocadas pelo seu Presidente ou pela maioria de seus membros;
    \item A Diretoria Executiva somente deliberará com a presença de, no mínimo, 3 (três) de seus membros, cabendo ao Diretor Presidente o voto de qualidade, no caso de empate.
    \item De cada reunião da Diretoria Executiva lavrar-se-á ata em livro próprio, assinada pelos Diretores presentes. 
    \end{enumerate}

\item Compete a qualquer Diretor, no âmbito de sus específicas atribuições e em conjunto com o Diretor Presidente, a prática de atos de gestão necessários ao funcionamento regular da Companhia, assim como:
        \begin{enumerate}[label=\roman*.]
        \item contratar, transigir e contrair obrigações em nome da Companhia;
        \item adquirir, onerar e alienar, a qualquer título, bens imóveis ou direitos a eles relativos, mediante prévia autorização do Conselho de Administração.
        \end{enumerate}

\item A Companhia só estará obrigada para com seus terceiros mediante assinatura de 2 (dois) Diretores, um dos quais será o Diretor Presidente, ou de 1 (um) Diretor e 1 (um) procurador especialmente nomeado.
        \begin{enumerate}[label= \S \arabic*]
        \item Na constituição de Procuradores ad negotia é indispensável a assinatura de 2 (dois) Diretores, sendo um dele o Diretor Presidente.
        \item Exceção será feita ao caso de poderes outorgados para representação em juízo, todas as procurações concedidas por tempo determinado, e terão vedadas o substabelecimento.
        \item A Companhia manterá um livro especial onde serão registradas todas as procurações outorgadas em seus nome e o teor das mesmas.
        \item No caso de necessidade de se praticarem atos no exterior, dos quais decorram obrigações à Companhia, estes poderão ser praticados por apenas 1 (um) diretor, desde que referendado previamente por 2 (dois) Diretores, um dos quais será o Diretor Presidente.
        \end{enumerate}

\item Compete à Diretoria Executiva, ouvido o Conselho de Administração, quando couber, e atendida a orientação geral e as diretrizes básicas traçadas pelo referido Conselho, a gestão dos negócios da Companhia, especialmente:

        \begin{enumerate}[label=\roman*.]
        \item estabelecer programa de atuação com vistas a consecução dos objetivos sociais, na conformidade da orientação fixada pelo acionista controlador, consoante com o art. 238 da Lei nº 6.404, de 15 de dezembro de 1976, pelo Conselho de Administração, pelas normas estatutárias e pelas deliberações da Assembleia Geral;
        \item aprovar no seu âmbito o Regimento Interno da Companhia, por proposta do Diretor Presidente, encaminhando-o ao Conselho de Administração;
        \item aprovar normas gerais para melhorar o desenvolvimento das atividades da Companhia;
        autorizar a celebração de quaisquer contratos cujo valor do principal torne exigível licitação, na modalidade de concorrência, bem como a de termos aditivos independente do valor;
        \item aprovar normas para aquisição e alienação de bens imóveis;
        \item aprovar o Regulamento de Licitações e contratações da Companhia;
        \item aprovar o plano de contas e o sistema de custos;
        \item aprovar normas de operação do sistema de transporte público sobre trilhos ou guiados, para o cumprimento dos dispositivos legais e regulamentares;
        \item autorizae a cessão de uso de patentes, marcas ou insígnias;
        \item autorizar a aquisição, alienação, locação, cessão ou oneração de bens móveis ou imóveis;
        \item autorizar a celebração de convênios e contratos, observando-se o disposto no inciso XV do artigo 18º desse Estatuto;
        \item autorizar a contratação de seguros, obras, serviços, estudos, projetos, pesquisas, empréstimos e financiamentos;
        \item autorizar atos judiciais ou extrajudiciais de renúncia, composição ou transação;
        \item autorizar a prática de atos no exterior, dos quais decorram obrigações para a Companhia;
        \item apresentar ao Conselho de Administração o balanço patrimonial, as demonstrações no exercício e as demonstrações das origens e aplicações de recursos, com os correspondentes pareceres do Conselho Fiscal e dos Auditores Independentes.
        \item autorizar despropriações;
        \item autorizar viagens no país e no exterior, de Diretores e empregados, por necessidade estrita do serviço, observada a legislação em vigor e as determinações do acionista controlador;
        \item autorizar a comercialização de bens imóveis de propriedade da Companhia; a propaganda e publicidade; a comercialização de tecnologia, serviços de consultoria, a manutenção de equipamentos de terceiros; o uso por terceiros, de áreas e espaços de propriedade da Companhia e a construção e implantação de sistemas de transporte e de terminais, no país e no exterior;
        \item autorizar a edição de jornais, revistas e outras publicações de caráter técnico ou comercial, de responsabilidade da Companhia;
        \item elaborar e submeter à aprovação do Conselho de Administração:
            \begin{enumerate}[label=\roman*.]
            \item as demonstrações financeiras e o relatório anual da Companhia;
            \item o orçamento da Companhia;
            \item o regulamento de seleção de forncedores de bens e serviços a serem contratados pela Companhia;
            \item a política de pessoal
            \end{enumerate}
        \item aprovar normas de remuneração de estudos, projetos, obras e serviços de qualquer natureza, vinculados ao objeto da Companhia;
        \item autorizar aditamentos não previstos em contratos;
        \item aprovar a participação da Companhia em eventos tais como congressos, seminários, feiras, etc;
        \item executar as tarefas que lhe forem delegadas;
        \item decidir sobre casos omissos, quando não forem competência do Conselho de Administração ou da Assembleia Geral;
        \end{enumerate}
\item Compete ao Diretor Presidente:
        
        \begin{enumerate}[label=\roman*.]
        \item representar a Companhia em conjunto com outro diretor, ativa e passivamente, em juízo ou fora dele, ressalvado os casos previstos nos parágrafos deste artigo;
        \item convocar e presidir as reuniões da Diretoria Executiva;
        \item articular-se com os órgãos públicos e privados, visando o conheciemento de planos, programas, projetos e respectivos financiamentos de sistemas de transporte, tráfego, trânsito e sistema viário;
        \item solicitar a manifestação do Conselho de Administração, sempre que julgar necessário;
        \item supervisionar, através do acompanhamento da atuação dos Diretores, as atividades de todas as unidades da Companhia;
        \item elaborar o Regimento Interno da Companhia, definindo a sua estrutura executiva, o seu funcionamento organizacional e as atribuições de seus membros detalhadamente;
        aproar o plano de cargos e salários e o quadro de pessoal da Companhia, de acordo com as necessidades administrativas e as condições existentes no mercado de trabalho, bem como as normas de admissão, provimento de função ou cargo de confiança, treinamento, segurança, medicina e higiene do trabalho, proteção do meio ambiente, direitos e deveres dos empregados;
        \item designar e nomear os ocupantes de cargo ou função de confiança da Companhia de acordo com o Plano de Cargos e Salários;
        \item propor à Diretoria as medidas de interesse da Companhia que dependam de sua aprovação;
        \item aprovar normas administrativas e técnicas, implementadoras das normas gerais aprovadas pela Diretoria;
        \item constituir procuradores, em conjunto com um dos Diretores, desde que autorizado pela Diretoria;
        \item autorizar admissões, transferências, reenquadramentos, promoções, remanejamentos, comissionamentos, alterações salariais, exonerações, punições e demissões de empregados, de acordo com as normas em vigor e nos limites do quadro de pessoal aprovado;
        \item autorizar a contratação de trabalhadores autônomos;
        \item autorizar a contratação de estagiários e trabalhadores temporários;
        \item emitir, endossar e avalizar letras de câmbio, duplicatas, notas promissórias, cheques e ordem de compra;
        \item receber e dar em quitação em conjunto com outro diretor;
        \item dispor sobre as substituições recíprocas entre os Diretores;
        \item assumir obrigações e firmar contratos de qualquer natureza, em conjunto com outro diretor, autorizado, quando necessário, pela Diretoria Executiva ou pelo Conselho de Administração, conforme o caso;
        \item autorizar a proposição de ações judiciais;
        \item constituir comissões, inclusive de sindicância e grupos de trabalho, mediante portaria;
        \item encaminhar ao Conselho de Administração, pelo menos semestralmente, relatórios a respeito do andamento dos negócios da Companhia;
        \item praticar atos de administração de pessoal no âmbito da Companhia, bem como aplicar penalidades disciplinares e, ainda, delegar, no todo ou em parte, quaisquer destas atribuições;
        \item promover através das Diretorias, os estudos técnicos necessários à captação de recursos externos e supervisionar sua aplicação;
        \item autorizar a abertura de licitação e homologar-lhe o resultado;
        \item emitir portarias e outros atos normativos de sua competência;
                \begin{enumerate}[label= \S \arabic*]
                \item É de competência exclusiva do Presidente a proposição, à Diretoria, do Regimento Interno;
                \item Os atos, contratos e instrumentos que acarretarem responsabilidades para a Companhia, especialmente os previstos nos incisos XIV, XV, XVI, e XVII deste artigo, serão praticados:
                        \begin{enumerate}[label=\roman*.]
                        \item pelo Presidente, em conjunto com outro Diretor;
                        \item por 2 (dois) Diretores, sendo um designado pelo Presidente;
                        \item pelo Presidente, ou Diretor por ele designado, e um Procurador;
                        \end{enumerate}
                \item O Presidente poderá designar um Diretor para receber citações, notificações, intimações e para representar a Companhia na lavratura de escrituras de desapropriação;
                \item O Diretor Presidente poderá, por intermédio de ato escrito, delegar poderes a qualquer dos Diretores, excetuando-se as suas competências exclusivas;
                \item A Auditoria Interna será subordinada diretamente ao Diretor Presidente, que deverá apreciar os seus resultados e sugestões, adotando medidas corretivas necessárias;
                \item O órgão de assessoramento jurídico será subordinado diretamente ao Presidente.
                \end{enumerate}
        \end{enumerate}
\item Os demais membros da Diretoria Executiva, a exceção do Diretor Presidente, terão suas atribuições definidas por áreas de negócio da sociedade, conforme dispuser o regimento interno da Companhia a ser aprovado pelo Conselho de Administração
\item Compete aos Diretores, além das funções definidas pelo Conselho de Administração e daquelas que lhe sejam conferidas pela Diretoria, ou pelo Presidente, desempenhar as seguintes atribuições:
        \begin{enumerate}[label=\roman*.]
        \item participar das reuniões de Diretoria, relatando os assuntos das respectivas áreas de coordenação e deliberando sobre a matéria em pauta;
        \item gerir as atividades da área da Companhia para a qual estiver designado, praticando os atos administrativos necessários, ressalvado o disposto em lei;
        \item executar as disposições constantes do Estatuto Social e as deliberações da Diretoria Executiva, do Conselho de Administração e da Assembleia Geral, no que se refere à sua área de atuação;
        \item propor ao Diretor Presidente a aplicação de penas ou sanções disciplinares;
        \item prestar assessoramento ao Diretor Presidente
        \end{enumerate}
\item É expressamente vedado e nulo, em relação à Companhia, o uso da denominação social em negócios estranhos aos seus interesses e objetivos, tais como a concessão de avais, fianças ou quaisquer outras garantias de favor.

\section{CONSELHO FISCAL}

\item O Conselho Fiscal, com as atribuições e poderes que a lei lhe confere, compor-se-á de 3 (três) membros efetivos e igual número de suplentes, residentes no país e eleitos, anualmente, pela Assembléia Geral Ordinária, sendo permitida a reeleição.

Parágrafo único. Dentre os membros do Conselho Fiscal eleitos pela Assembléia Geral Ordinária, 1 (um) será indicado pelo Estado de Minas Gerais, 1 (um) pelo Município de Belo Horizonte e 1 (um) pelo Município de Contagem.

\item O Conselho Fiscal terá funcionamento permanente e deverá reunir-se mensalmente, independentemente de convocação.

\item Ao Conselho Fiscal compete:
        \begin{enumerate}[label=\roman*.]
        \item fiscalizar os atos da Diretoria Executiva e verificar o cumprimento de seus deveres legais e estatutários;
        \item opinar sobre o relatório anual da administração, fazendo constar de seu parecer as informações complementares que julgar necessárias ou úteis à deliberação da Assembléia Geral.
        \item opinar sobre as propostas dos órgãos da administração, a serem submetidas à Assembléia Geral, relativas à modificação do capital social, planos de investimentos ou orçamento de capital, distribuição de dividendos, transformação, incorporação, fusão ou cisão.
        \item denunciar aos órgãos da administração e, se estes não tomarem providências necessárias para a proteção dos interesses da Companhia, à Assembléia Geral, os erros, fraudes ou crimes que descobrirem, caso aconteçam, e sugerir providências à Companhia.
        \item convocar a Assembléia Geral Ordinária, se or órgãos da administração retardarem por mais de 1 (um) mês esta convocação, e na Extraordinária, sempre que ocorrerem motivos graves ou urgentes, incluindo na agenda da Assembléia as matérias que considerarem necessárias;
        \item analisar, as menos trimestralmente, o balancete e demais demonstrações financeiras elaboradas periodicamente pela Companhia;
        \item examinar as demonstrações financeiras do exercício social e sobre elas opinar;
        \end{enumerate}

\item Os membros efetivos do Conselho Fiscal farão jus à uma remuneração fixada pela Assembléia Geral.

\chapter{EXERCÍCIO SOCIAL, BALANÇO GERAL, RESERVA E DIVIDENDOS}

\item O exercício social coincide com o ano civil, e, a 31 d dezembro de cada ano, proceder-se-á o Balanço Patrimonial e demais demonstrações financeiras, observadas as prescrições legais pertinentes.

\item Os lucrso líquidos do exercício tem a seguinte destinação:

        \begin{enumerate}[label=\roman*.]
        \item 5\% (cinco por cento) para constituição do fundo de reserva legal, sendo que esta dedução deixa de ser necessária tão logo este fundo atinja 20\% (vinte por cento) da cifra que representa o capital social.
        \item 6\% (seis por cento) para distribuição de um dividendo mínimo obrigatório;
        \item o saldo para a constituição de uma reserva especial para o aumeto de capital, facultado à Assmebléia Geral, mediante proposta da Diretoria, apropriar parte ou totalidade desse saldo para distribuição suplementar de dividendos, ou constituição de reservas técnicas legalmente admissíveis.
        \end{enumerate}

\item Os dividendos não reclamados não renderão juros e, ao fim de 3 (três) anos, prescreverão em favor da Companhia.

\item Em caso de eventuais déficits operacionais, os memos serão cobertos pelos acionistas, na mesma proporção da sua participação no capital da Companhia.

\chapter{DOS EMPREGADOS DA COMPANHIA}

\item O regime jurídico dos empregados da Compahia será o da Legislação Trabalhista.

\item A admissão na Companhia será realizada mediante aprovação em concurso público, nos níveis salariais iniciais de cada cargo.

Parágrafo único. Além da admissão de pessoal na forma prevista no caput deste artigo poderá haver, pela Companhia, a admissão de pessoal oriundo da Companhia Brasileira de Trens Urbanos - CBTU, baseada, em cada caso, nos estudos técnicos realizados especialmente para esse fim, nos termos da legislação pertinente.

\item O empregado só poderá ser cedido para órgãos federais, estaduais ou municipais e sociedades de economia mista se a cessionária reembolsar a cedente do valor da remuneração do empregado, acrescida dos respectivos encargos.

Parágrafo único. As cessões não poderão ultrapassar o período de 2 (dois) anos admitida somente uma renovação, envolvendo apenas empregados do quadro permanente.

\chapter{DISPOSIÇÕES FINAIS E TRANSITÓRIAS}

\item Na integralização inicial do capital da Companhia, as ações serão subscritas pelo Estado e pelos Municípios, de forma a se obetr a seguinte proporção do capital votante: Estado de Minas Gerais com 55\% (cinquenta e cinco por cento), Município de Belo Horizonte 35\% (trinta e cinco por cento) e Município de Contagem 10\% (dez por cento).

\item Os casos omissos neste Estatuto serão resolvidos pela Assembléia Geral.

\item Para todos os efeitos passam a integrar este Estatuto, no que foram aplicáveis, as disposições da legislação estadual pertinentes às empresas descentralizadas.

\end{enumerate}

\end{document}
